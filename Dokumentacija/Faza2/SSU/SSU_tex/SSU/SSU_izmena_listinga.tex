\documentclass[12pt]{article}
\usepackage[T1]{fontenc}
\usepackage[utf8]{inputenc}
\usepackage[english,serbian]{babel}
\usepackage[nottoc]{tocbibind}
\usepackage[subsection]{algorithm}
\usepackage{caption}
\usepackage[noend]{algpseudocode}
\usepackage[T1,T2A]{fontenc}
\usepackage[utf8]{inputenc}
\usepackage{tikz}
\usepackage{pgfplots}
\usepackage{float}
\usepackage{xfrac}
\usepackage{amssymb, amsmath, amsthm}
\usepackage{caption, subcaption}
\usepackage{fancyhdr}
\usepackage{geometry}
\usepackage{xcolor}
\usepackage{indentfirst}
\usepackage[utf8]{inputenc}
\usepackage{tikz}
\usepackage{url}
\usepackage{listings}
\usepackage{graphicx}
\usepackage{tabularx}
\usepackage{textcomp}
\graphicspath{ {./images/} }
% References
% https://www.overleaf.com/learn/latex/Bibliography_management_in_LaTeX
% https://en.wikibooks.org/wiki/LaTeX/Bibliography_Management

\begin{document}
    \selectlanguage{serbian}
    \renewcommand{\labelenumii}{\arabic{enumi}.\arabic{enumii}}
	\begin{titlepage}  
		\center
		\textbf{ \LARGE ELEKTROTEHNIČKI FAKULTET, UNIVERZITET U BEOGRADU } \\[4cm]
		\textbf{ \Large PROJEKAT ĐIRO\texttrademark} \\[0.3cm]
		iz predmeta \\[0.3cm]
		\textbf{ \Large Principi softverskog inženjerstva} \\[0.7cm]
		{ \huge \bfseries Specifikacija scenarija upotrebe funkcionalnosti izmene listinga } \\[6cm]
		

		\begin{minipage}{0.5\textwidth}
			\begin{flushleft}
				\large
				\emph{Tim} SLAV Co. \\
			     $\;\;\; \cdot \;\;$Stefan Branković  0253/2019\\
			     $\;\;\; \cdot \;\;$Lazar Erić 0235/2019\\
			     $\;\;\; \cdot \;\;$Aleksa Račić 728/2019\\
			     $\;\;\; \cdot \;\;$Vasilevska Nevena 0418/2019
			\end{flushleft}
		\end{minipage}
		~
		\begin{minipage}{0.4\textwidth}
			\begin{flushright}
				\large
				\emph{Verzija:} \\
				1.0
			\end{flushright}
		\end{minipage} \\[2cm]
		\enlargethispage{4\baselineskip}
		{ \large Beograd, mart 2021. }
		\vfill
	\end{titlepage}
\pagebreak
\tableofcontents
\pagebreak



\section{Uvod}
\subsection{Rezime}
Definisanje scenarija upotrebe pri izmeni listinga, sa primerima odgovarajućih html stranica
\subsection{Namena dokumenta i ciljne grupe}
Dokument će koristiti svi članovi projektnog tima u razvoju projekta i testiranju a može se koristiti i pri pisanju uputstva za
upotrebu.
\subsection{Reference}
\begin{enumerate}
   \item Projektni zadatak
   \item Uputstvo za pisanje specifikacije scenarija upotrebe funkcionalnosti
   \item  Guidelines – Use Case, Rational Unified Process 2000
   \item  Guidelines – Use Case Storyboard, Rational Unified Process 2000
 \end{enumerate}
\subsection{Otvorena pitanja}


\begin{center}
\begin{tabular}{ | m{2cm} | m{7cm}| m{7cm} | } 
\hline
Redni broj& Opis & Rešenje \\ 
\hline
1 & Koje informacije o vozilu treba dozvoliti da se menjaju posle kreiranja naloga? & \\
\hline
2 & Koje informacije treba zabraniti da s emenjaju kada za dato vozilo postoji rezervacija? & \\
\hline
\end{tabular}
\end{center}

    

\section{Scenario izmene listinga}
\subsection{Kratak opis}
Treba da omogući dodavanje novih ili izmenu postojećcih informacija na listingu.
Korisnici sa rezervacijom za izmenjeni automobil treba da budu obavešteni o izmeni.
Informacije koje se ne mogu izmeniti su informacije koje se odnose na proizvodača,
model, godište automobila i slične.
\subsection{Tok događaja}

\subsubsection{Uspese izmene na listingu}
\begin{enumerate}
    \item Korisnik unosi dodatne ili menja postojeće informacije o vozilu, model, proizvodjaca, godina proivodnje itd.
    \item Kada završi korisnik sve izmene pritiska dugme sačuvaj.
    \item Ako su mu dozvoljene sve promene, vraća se na početnu stranu.
    \item Klijenti koji imaju rezervacije tog vozila dobijaju informacije o promenama.
\end{enumerate}

\subsubsection{Neuspešne izmene na listingu}
\begin{enumerate}
    \item Korisnik unosi dodatne ili menja postojeće informacije o vozilu, model, proizvodjaca, godina proivodnje itd.
    \item Kada završi korisnik sve izmene pritiska dugme sačuvaj.
    \item Pošto ima promene koje nisu dozvoljene (sama provera da li su dozvoljenje se vidi tek kada se stisne dugme za čuvanje, jer mora postojati promena nekih statičkih karakteristika u slučaju da je pogrešno prvi put napisano), ako ima već rezervacije na primer ili menja podatke koji ne smeju da se menjaju(model, godište itd.).
    \item Korisnik dobija informaciju da j+ne sme da menja informaciju i vraća se na tačku 1.
\end{enumerate}

\subsection{Posebni zahtevi}
Nema ih.
\subsection{Preduslovi}
Korinsik koji želi da izmeni listing mora da bude ulogovan i naravno mora postojati sam listing da bi se mogla izvšiti sama izmena.
\subsection{Posledice}
Izmene na oglasu koje je korisnik postavio i obaštenje o tome klijentima koji trenutno imaju rezervacije tog automobila rezervacije.

\section{Istorija izmena}
\begin{center}
\begin{tabular}{ | m{2cm} | m{1.5cm}| m{6cm} | m{5cm} | } 
\hline
Datum & Verzija & Kratak opis & Autori \\ 
\hline
 21.03.2022. & 1.0 & Napravljen inicijalni dokument & Nevena Vasilevska\\ 
\hline
&&&\\ 
\hline
\end{tabular}
\end{center}
\end{document}
